% !TEX TS-program = pdflatexmk

% arXiver will snag the right plots so they show up on Twitter :)
%@arxiver{chi-eff-distributions.pdf,Wills_evidence_ratio_figure_with_mixture_models.png,chi-eff-mock-posteriors.pdf,six-way-O2-model-selection.pdf}

\documentclass[modern,linenumbers]{aastex61}

\usepackage{acronym}
\usepackage{amsmath}
\usepackage{amssymb}

\newcommand{\chieff}{\chi_\mathrm{eff}}
\newcommand{\dd}{\mathrm{d}}
\newcommand{\diff}[2]{\frac{\dd #1}{\dd #2}}
\newcommand{\fixme}[1]{\textcolor{red}{FIXME: #1}}
\newcommand{\pali}{p_\mathrm{ali}}
\newcommand{\piso}{p_\mathrm{iso}}

\newcommand{\checkme}[1]{\textcolor{red}{#1}}
%\newcommand{\checkme}[1]{ #1 }

\newcommand{\OOneSigmaIsoAligned}{\checkme{1.7}}
\newcommand{\OOneOddsIsoAligned}{\checkme{0.087}}
\newcommand{\OTwoSigmaIsoAlignedMin}{\checkme{2.9}}
\newcommand{\OTwoOddsIsoAlignedMin}{\checkme{0.0035}}

\newcommand{\ilya}[1]{\textcolor{purple}{#1}}
\newcommand{\will}[1]{\textcolor{cyan}{#1}}

\begin{document}

\acrodef{BBH}{binary black hole}
\acrodef{BH}{black hole}
\acrodef{EM}{electromagnetic}
\acrodef{GW}{gravitational wave}
\acrodef{PE}{parameter estimation}

\title{Distinguishing Spin-Aligned and Isotropic Black Hole
  Populations With Gravitational Waves}

\author[0000-0003-1540-8562]{Will M. Farr}

\affiliation{Birmingham Institute for Gravitational Wave Astronomy and
  School of Physics and Astronomy, University of Birmingham,
  Birmingham, B15 2TT, United Kingdom}

\author[0000-0002-6100-537X]{Simon Stevenson}

\affiliation{Birmingham Institute for Gravitational Wave Astronomy and
  School of Physics and Astronomy, University of Birmingham,
  Birmingham, B15 2TT, United Kingdom}
\affiliation{Kavli Institute of Theoretical Physics, University of California, Santa Barbara}

\author{M. Coleman Miller}

\affiliation{Department of Astronomy and Joint Space-Science
  Institute, University of Maryland, College Park, MD 20742--2421,
  United States}

\author[0000-0002-6134-8946]{Ilya Mandel}

\affiliation{Birmingham Institute for Gravitational Wave Astronomy and
  School of Physics and Astronomy, University of Birmingham,
  Birmingham, B15 2TT, United Kingdom}
 \affiliation{Kavli Institute of Theoretical Physics, University of California, Santa Barbara}

\author[0000-0002-6254-1617]{Alberto Vecchio}

\affiliation{Birmingham Institute for Gravitational Wave Astronomy and
  School of Physics and Astronomy, University of Birmingham,
  Birmingham, B15 2TT, United Kingdom}

\email{w.farr@bham.ac.uk,simons@star.sr.bham.ac.uk,miller@astro.umd.edu,imandel@star.sr.bham.ac.uk,av@star.sr.bham.ac.uk}

\begin{abstract}
  The first direct detections of \acp{GW} from merging \acp{BBH} open
  a unique window into the formation environment of massive stars and
  their compact remnants.  One promising signature of the formation
  environment is the angular distribution of the \ac{BH} spins;
  systems formed through dynamical interactions among already-compact
  objects are expected to have isotropic spin orientations whereas
  binaries formed from pairs of stars born together are more likely to
  have spins preferentially aligned with the binary orbit as a
  consequence of their joint evolution toward a \ac{BBH} system.  We
  consider existing \ac{GW} measurements of $\chieff$, the
  best-measured combination of spin parameters, in the three likely
  binary black hole detections GW150914, LVT151012, and GW151226.  If
  binary black hole spin magnitudes extend to high values, as is
  suggested by observations of black hole X-ray binaries, we show that
  the data already exhibit a $\OOneSigmaIsoAligned\sigma$
  ($\OOneOddsIsoAligned$ odds ratio\footnote{An odds ratio of $r$ with
    $r \ll 1$ is equivalent to $x \sigma$ with
    $x = \Phi^{-1}\left( 1 - r \right)$, where $\Phi$ is the unit
    normal CDF.})  preference for an isotropic angular distribution.
  By considering the effect of an additional 10 detections drawn from
  the various models in the suite we show that if all observations
  come from a single population such an augmented data set would
  enable at least a $\OTwoSigmaIsoAlignedMin\sigma$
  ($\OTwoOddsIsoAlignedMin$ odds ratio) distinction between the
  isotropic and aligned models for the assumed spin magnitude
  distributions, and in most cases better than $5\sigma$
  ($2.9 \times 10^{-7}$ odds ratio).  The existing preference for
  either an isotropic spin distribution or low spin magnitudes for the
  observed systems will be confirmed (or overturned) confidently in
  the near future by subsequent observations.
\end{abstract}

\acresetall{}

\section{\ac{GW} Spin Measurements and Model Selection}
\label{sec:O1}

Following the detection of a merging \ac{BBH} system, \ac{PE} tools
\citep{2015PhRvD..91d2003V} compare model gravitational waveforms
\citep[e.g.\
][]{2014PhRvD..89h4006P,2014PhRvD..89f1502T,2014PhRvL.113o1101H}
against the observed data to obtain a posterior distribution on the
parameters that describe the compact binary source.

The spin parameter with the largest effect on waveforms, and a
correspondingly tight constraint from the data, is a mass-weighted,
aligned, ``effective spin,''
\begin{equation}
  \chieff = \frac{c}{GM} \left( \frac{\vec{S}_1}{m_1} + \frac{\vec{S}_2}{m_2}
  \right) \cdot \frac{\vec{L}}{\left| \vec{L} \right|} = \frac{1}{M} \left( m_1 \chi_1 + m_2 \chi_2 \right),
\end{equation}
where the $\hat{z}$ axis is aligned with the orbital angular momentum
of the binary, $m_{1,2}$ are the masses of the more-massive (1) and
less-massive (2) components, $M = m_1 + m_2$ is the total mass,
$\vec{S}_{1,2}$ are the spin angular momentum vectors of the black
holes in the binary, $\vec{L}$ is the orbital angular momentum vector,
and $0 \leq \chi_{1,2} \leq 1$ are the corresponding dimensionless
projections of the individual \ac{BH} spins
\citep{2016PhRvL.116x1102A}.

Figure \ref{fig:O1-posteriors} shows an approximation to the posterior
inferred on $\chieff$ for the three likely \ac{GW} detections
GW150914, GW151226 and LVT151012 from \citet{O1-BBH}.  Because samples
drawn from the posterior on $\chieff$ are not publicly released at
this time, we have approximated the posterior as a Gaussian
distribution with the same mean and 90\% credible interval as quoted
in \citet{O1-BBH}.  There is essentially no posterior support for
$\chi \gtrsim 0.5$ as observed in the majority of the
electromagnetically-detected stellar-mass black hole population (see
Section \ref{sec:discussion}).  The analysis here is relatively
insensitive to the precise details of the posterior distributions;
other conclusions are more sensitive.  In particular, our
approximation does permit $\chieff = 0$ for GW151226 while the true
posterior rules this out at high confidence
\citep{2016PhRvL.116x1103A,O1-BBH}.

\begin{figure}
  \plotone{../plots/chi-eff-mock-posteriors}
  \caption{\label{fig:O1-posteriors} Approximate posteriors on
    $\chieff$ from the LIGO O1 observations in \citet{O1-BBH}.  We
    approximate the posteriors reported in \citet{O1-BBH} using
    Gaussians with the same median and 90\% credible interval as
    reported in \citet{O1-BBH}.  It is notable that none of the
    $\chieff$ posteriors extends to the high spin magnitudes inferred
    from several electromagnetically-detected stellar-mass black hole
    systems (see \citet{2015PhR...548....1M} for a summary of such
    measurements).}
\end{figure}

Small values of $\chieff$ as exhibited in these systems can result
from either intrinsically small spins or larger spins whose direction
is mis-aligned with the orbital plane of the binary (i.e.\ spin
vectors with small $z$-components).  Mis-alignment is capable of
producing \emph{negative} values of $\chieff$, however, while aligned
spins will always have $\chieff \geq 0$.  This difference provides
strong discriminating power between the two angular distributions,
even without good information about the magnitude distribution; to the
extent that data favour negative $\chieff$ they weigh heavily against
aligned models.  To quantify the degree of support for these two
alternate explanations of small $\chieff$ values in the merging
\ac{BBH} population, we compared the Bayesian evidence for various
simple models of the spin population using the \ac{GW} data set.

Each of our models for the merging \ac{BBH} spin population assumes
that the merging black holes are of equal mass (this is marginally
consistent with the three observations \citep{O1-BBH}, and the
$\chieff$ distribution is not particularly sensitive to the mass ratio
between the merging objects---see Section \ref{sec:mass-ratio}).  We
assume that the population spin distribution factorises into a
distribution for the spin magnitude $a$ and a distribution for the
spin angles.  Finally, we assume that the distribution of spins is
common to each component in a merging binary.  Choosing one of three
magnitude distributions
\begin{equation}
  \label{eq:magnitude-dists}
  p(a) = \begin{cases}
    2\left(1-a\right) & \textnormal{``low''} \\
    1 & \textnormal{``flat''} \\
    2 a & \textnormal{``high''}
  \end{cases},
\end{equation}
and pairing with either an isotropic angular distribution or a
distribution that generates perfect alignment with the positive $z$
axis yields six different models for the $\chieff$ distribution.
These models are shown in Figure \ref{fig:chieff-distribution-models}.

\begin{figure}
  \plotone{../plots/chi-eff-distributions}
  \caption{\label{fig:chieff-distribution-models} The models for the
    distribution of $\chieff$ considered in this paper.  In all models
    we assume that the binary mass ratio $q \equiv m_1/m_2 = 1$ and
    that the distribution of spin vectors is the same for each
    component.  The ``flat'' (blue lines), ``high,'' (green lines),
    and ``low'' (red lines) magnitude distributions are defined in
    Eq.\ \eqref{eq:magnitude-dists}.  Solid lines give the $\chieff$
    distribution under the assumption that the orientations of the
    spins are isotropic; dashed lines give the distribution under the
    assumption that both objects' spins are aligned with the orbital
    angular momentum.  The isotropic distributions are readily
    distinguished from the aligned by the production of negative
    $\chieff$ values, while the distinction between the three models
    for the spin magnitude distribution is less sharp.}
\end{figure}

The distributions in Eq.\ \eqref{eq:magnitude-dists} are not meant to
represent any particular physical model, but rather to capture our
uncertainty about the spin magnitude distribution, discussed in detail
in Section \ref{sec:discussion}; neither observations nor population
synthesis codes can at this point authoritatively suggest {\it any}
particular spin distribution \citep{2015PhR...548....1M}.  Our models,
however, allow us to see how sensitive the $\chieff$ distribution is
to spin alignment given uncertainties about the spin magnitudes.

We fit hierarchical models of the three LIGO O1 observations using
these six different, zero-parameter population distributions (see
Section \ref{sec:hierarchical}).  We also fit three mixture models for
the population, where the spin magnitude distribution is fixed but the
angular distribution is a weighted sum of the isotropic and aligned
distribution.  The evidence, or marginal likelihood, for each of the
models is shown in Figure \ref{fig:O1-odds}.  For all three magnitude
distributions, the mixture models' posterior on the mixing fraction
peaks at 100\% isotropic, which explains why the zero-parameter,
pure-isotropic models are preferred over the single-parameter mixture
models for every magnitude distribution with this data set.  Not
surprisingly, given the small $\chieff$ values in the three detected
systems, the most-favoured model among those with an isotropic angular
distribution has the ``low'' magnitude distribution; the most favoured
model among those with an aligned distribution also has the ``low''
magnitude distribution.  The odds ratio between the ``low'' aligned
and ``low'' isotropic models is $\OOneOddsIsoAligned$, or
$\OOneSigmaIsoAligned\sigma$; thus the data favour isotropic spins
among our suite of models.  While the data favour spin amplitude
distributions with small spin magnitudes, note that a model with all
\ac{BBH} systems having zero spin is ruled out by the GW151226
measurements, which bound $\chieff \gtrsim 0.2$ at 90\% credibility
\citep{2016PhRvL.116x1103A}.

\begin{figure}
  \plotone{../plots/Wills_evidence_ratio_figure_with_mixture_models}
  \caption{Odds ratios among our models using the approximations to
    the posteriors on $\chieff$ from the O1 observations shown in
    Figure \ref{fig:O1-posteriors}.  The flat (``F''), high (``H''),
    and low (``L'') spin magnitude distributions (see Eq.\
    \eqref{eq:magnitude-dists}) are paired with isotropic (``I'') and
    aligned (``A'') angular distributions, as well as a mixture model
    of the two (``$\mathrm{M}$'').  The most-favoured models have the
    ``low'' distribution of spin magnitudes.  The odds ratio between
    these models is $\OOneOddsIsoAligned$, or
    $\OOneSigmaIsoAligned\sigma$.  For all magnitude distributions the
    pure-isotropic models are preferred over the mixture models;
    correspondingly, the posterior on the mixture fraction peaks at
    100\% isotropic.}
  \label{fig:O1-odds}
\end{figure}

\subsection{Future Spin Measurements}
\label{subsec:future}

Estimates of the rate of \ac{BBH} coalescences give a reasonable
chance of 10 additional \ac{BBH} detections in the next two years
\citep{O1-BBH,2016ApJ...833L...1A,2016ApJS..227...14A}.  Assuming an
additional 10 detections with similar observational uncertainties
drawn from each of our six zero-parameter models for the spin
distribution in addition to the three existing detections from O1, we
find the odds ratios shown in Figure \ref{fig:O2-predictions}.  We
find that most scenarios with an additional 10 detections allow the
correct angular distribution to be distinguished with greater than
$5\sigma$ ($2.9 \times 10^{-7}$ odds) credibility; and in the most
pessimistic case the distinction is typically
$\OTwoSigmaIsoAlignedMin\sigma$ ($\OTwoOddsIsoAlignedMin$ odds ratio).
While such future detections should permit a confident distinction
between \emph{angular} distributions, we will remain much less certain
about the \emph{magnitude} distribution until we have a larger number
of observations.  In Figure \ref{fig:O2-predictions}, the odds ratio
between different magnitude distributions with the same angular
distribution is much closer to one than the odds ratio between angular
distributions.

\begin{figure}
  \plotone{../plots/six-way-O2-model-selection}
  \caption{\label{fig:O2-predictions} Distribution of odds ratios
    predicted with 10 additional observations above the three
    discussed in Section \ref{sec:O1}.  Each panel corresponds to
    additional observations drawn from one of the $\chieff$
    distribution models.  The model from which the additional
    observations are drawn is outlined in red.  The height of the blue
    bar gives the median odds ratio relative to the model from which
    the additional observations are drawn; the green line gives the
    68\% ($1 \sigma$) symmetric interval of odds ratios over 1000
    separate draws from the model distribution.  The closest median
    ratio between the most-favoured isotropic model and the
    most-favoured aligned model is $\OTwoOddsIsoAlignedMin$,
    corresponding to $\OTwoSigmaIsoAlignedMin\sigma$ preference for
    the correct angular distribution; most models result in more than
    $5\sigma$ preference for the correct angular distribution.
    Because the three observations from Section \ref{sec:O1} are
    included in each data set the ``correct'' model is not necessarily
    preferred over the others, particularly when that model uses the
    ``high'' magnitude distribution, which is strongly dis-favoured
    from the O1 observations alone.}
\end{figure}

\section{Discussion}
\label{sec:discussion}

Most of our resolving power for the spin angular distribution is a
result of the fact that our ``aligned'' models cannot produce
$\chieff < 0$ (see Figure \ref{fig:chieff-distribution-models}).  If
spins are intrinsically very small, with $a \lesssim 0.3$, then it is
no longer possible to resolve the negative spin with a small number of
observations.  As noted below, however, spins observed in X-ray
binaries are typically large.  Additionally, models which do not
permit \emph{some} spins with $\chieff \gtrsim 0.1$ are ruled out by
the GW151226 observations \citep{2016PhRvL.116x1103A}.

In order to perform our analysis we need to select models for the
distribution of the spin magnitudes of stellar-mass black holes.
Observational data for such model selection is sparse.
\citet{2015PhR...548....1M} give current estimates of the spin
parameters for stellar-mass black holes, obtained using disk
reflection and/or disk continuum methods.  Most of the systems studied
are low-mass X-ray binaries rather than the high-mass X-ray binaries
that are likely to be the progenitors of double black hole binaries.
In addition, there are substantial systematic errors that can
complicate either type of analysis (see \citealt{2015PhR...548....1M}
for a discussion), and selection effects could yield a biased
distribution. Nonetheless, if we take the reported spin magnitudes as
representative then we find that there is a preference for high spins;
for example, 14 of the 19 systems with reported spins have
dimensionless spin parameters in excess of 0.5.  It is usually argued
that the masses and spin parameters of stellar-mass black holes are
unlikely to be altered significantly by accretion (low-mass donors may
not have enough mass and high-mass donors have a very short phase in
which they transfer mass; a variant of this long-standing argument was
presented by \citet{1999MNRAS.305..654K}), but see
\citet{2003MNRAS.341..385P,2015ApJ...800...17F}.  Thus the current
spin parameters probably are close to their values upon core collapse.
However, the specific processes involved in the production of black
hole binaries from isolated binaries could alter the spin magnitude
distribution of those holes relative to the X-ray binary systems; for
example, close tidal interactions could spin up the core, or stripping
of the envelope could reduce the available angular momentum
\citep{2016MNRAS.462..844K,2017arXiv170200885Z,2017arXiv170203952H}.

The spin directions of binary black holes formed dynamically through
interactions in dense stellar environments
\citep{SigurdssonHernquist:1993,PZMcMillan:2000,Rodriguez:2015,Stone:2016}
are expected to be isotropic given the absence of a preferred
direction \citep[e.g.,][]{2016ApJ...832L...2R} and the persistence of
an isotropic distribution through post-Newtonian evolution
\citep{2004PhRvD..70l4020S,2007ApJ...661L.147B}.

The spin directions in isolated binaries, whether evolving via the
classical channel through a common-envelope phase
\citep{TutukovYungelson:1973,TutukovYungelson:1993,Lipunov:1997,2016Natur.534..512B,Stevenson:2017}
or through chemically homogeneous evolution
\citep{MandeldeMink:2016,Marchant:2016} are generally expected to be
preferentially aligned.  Despite observed spin-orbit misalignments in
both massive stellar binaries
\citep{Albrecht:2009,2014ApJ...785...83A} and BH XRBs
\citep{Orosz:2001,Martin:2008b,Martin:2008,MorningstarMiller:2014},
mass accretion and tidal interactions will tend to realign the binary.
On the other hand, a supernova natal kick (if any) can change the
orbital plane and misalign the binary
\citep{2000ApJ...541..319K,2013PhRvD..87j4028G}; the supernova can
also tilt the spin angle, as in the double pulsar J0737-3039
\citep{2011ApJ...742...81F}; and a variety of uncertain processes,
such as wind-driven mass loss and post-collapse fallback, can couple
the spin magnitude and direction distributions, contrary to our
simplified assumptions.

\citet{2017CQGra..34cLT01V} also studied the possibility of
distinguishing aligned and isotropic angular distributions of \ac{BBH}
spins, but concluded that several hundred sources would be required to
adequately separate models which included both parallel and
anti-parallel spins for an ``aligned'' population.  In contrast, and
in agreement with the present study, \citet{Stevenson:2017spin} found
that only $\sim 5$ observations of rapidly spinning black holes would
be necessary to distinguish isotropic and aligned spin distributions,
though tens of detections would be required for more nuanced admixture
models.  Meanwhile, recent studies by
\citet{2017arXiv170306869F,2017arXiv170306223G} focused on inference
on spin magnitudes.

In summary, the angular distribution of spins in \acp{BBH} formed
through isolated binary evolution is uncertain; nevertheless, it is
more probable that spins are preferentially aligned after evolution
through this channel than not, while an isotropic distribution of
spins is the natural outcome of dynamical formation processes.
Therefore, if the current observational trend for low $\chieff$
continues with future gravitational-wave observations, it will be
possible to either confirm that \ac{BH} spins are sometimes large in
magnitude and isotropic in direction, yielding a strong indication of
a dynamical formation origin; or that \ac{BH} spins are overwhelmingly
small in magnitude, yielding a notable contradiction with the claimed
X-ray binary spin measurements, particularly those for high-mass X-ray
binaries.  

%\citet{2017CQGra..34cLT01V} also studied the possibility of distinguishing aligned and isotropic angular distributions of \ac{BBH} spins, but concluded that several hundred sources would be required to adequately separate the models.  The ``aligned'' population in that study permitted spins nearly parallel or anti-parallel to the orbital axis, i.e.\ $\hat{S}\cdot \hat{L} \simeq \pm 1$, in contrast to our aligned models; thus the aligned/isotropic discriminating power in \citet{2017CQGra..34cLT01V} relies effectively on measurements of the detailed shape of the $\chieff$ distribution and, as noted above, measurements are not very sensitive to the spin magnitude distribution that controls the shape of $\chieff$.  Note that anti-parallel spins are strongly disfavoured in models of isolated binary evolution (see, for example, \citet{Kalogera:2000}).


\acknowledgements

We thank Richard O'Shaughnessy and Christopher Berry for discussions and
comments on this work.  WF, SS, IM and AV were supported in part by the STFC.  MCM acknowledges support of the University of
Birmingham Institute for Advanced Study Distinguished Visiting Fellows
program.   SS and IM acknowledge support from the National Science Foundation under Grant No. NSF PHY11-25915.

\appendix

\section{Mixture model}

While we carried out Bayesian comparisons between isotropic and aligned spin distributions under various assumptions, a preference for one of the considered models over the others does not necessarily indicate that it is the correct model.  All of the considered models could be inaccurate for the actual distribution, especially since all of the considered models are based on a number of additional assumptions, such as decoupled spin magnitude and spin misalignment angle distributions and identical distributions for primary and secondary spins.

We now partly relax the simplified assumptions made earlier by considering the possibility that the true distribution of BBH spin-orbit misalignments observed by LIGO is a mixture of binaries with aligned spins and binaries with isotropic spins.

%
\begin{figure}
\centering
\includegraphics[width=0.45\textwidth]{../plots/posterior_on_isotropic_fraction.png}
\caption{\textbf{Fraction of the BBH population coming from an
    isotropic distribution under a mixture model.} The dotted line
  shows the flat prior on the fraction of BBHs coming from an
  isotropic distribution, $f_i$, under the mixture model. The 3 red
  lines show the posterior on $f_i$ after O1 with our various
  assumptions regarding BH spin magnitudes.  The solid line shows the
  posterior assuming that all BHs have their spin magnitude drawn from
  the ``flat'' distribution. The dashed line assumes the ``high'' BH
  spin magnitude distribution $p(a) = 2a$. The dot-dash line assumes
  the ``low'' distribution $p(a) = 2(1-a)$.  We see that for a wide
  range of assumptions regarding BH spin magnitudes, the fraction
  coming from an isotropic distribution $f_i$ peaks at 1.}
\label{fig:mixture_fraction_posterior}
\end{figure}
%

%-- could also use the cumulative posterior
%%
%\begin{figure}
%\centering
%\includegraphics[width=0.45\textwidth]{../plots/isotropic_fraction_cumulative_posterior.png}
%\caption{\textbf{Fraction of the BBH population coming from an isotropic distribution} The dotted (blue) line shows the cumulative distribution for the flat prior on the fraction of BBHs coming from an isotropic distribution $f_i$. The solid (red) line shows the cumulative distribution for the posterior after O1, assuming that all BHs have their spin magnitude drawn from a uniform distribution. The dashed line assumes a `increasing' distribution $p(a) = 2a$ for BH spin magnitudes, whilst the dot-dash line assumes a `decreasing' distribution $p(a) = 2(1-a)$. We see that regardless of our assumption regarding BH spin magnitudes, there is a preference for a large fraction coming from an isotropic distribution.}
%\label{fig:mixture_fraction_posterior}
%\end{figure}
%

Following \citet{Stevenson:2017spin}, we fit a mixture model (labelled
model 'M' in Figure~\ref{fig:O1-odds}) where a fraction $f_i$ of BBHs
have spins drawn from an isotropic distribution, whilst a fraction
$1 - f_i$ have their spins aligned with the orbital angular
momentum. We assume a flat prior on the fraction $f_i$. To test the
robustness of our result, we vary the distribution we assume for BH
spin magnitude distributions as with the aligned and isotropic
models. We use the ``flat'', ``high'' and ``low'' distributions
(Equation~\ref{eq:magnitude-dists}), assuming all BHs have their spin
magnitude drawn from the same distribution for both the aligned and
isotropic populations. We calculate and plot the posterior on $f_i$
given by Equation~\ref{eq:hierarchical-posterior} ($f_i=\lambda$ in
the derivation) in Figure~\ref{fig:mixture_fraction_posterior}. We
find the mean fraction of BBHs coming from an isotropic distribution
is 0.63, 0.71 and 0.78 assuming the ``low'', ``flat'' and ``high''
distributions for spin magnitudes respectively, compared to the prior
mean of 0.5. The lower 90\% limits are 0.26, 0.39 and 0.52
respectively, compared to the prior of 0.1. In all cases, the
posterior peaks at $f_i = 1$. Thus, for these spin magnitude
distributions we find that the current O1 LIGO observations constrain
the majority of BBHs to have their spins drawn from an isotropic
distribution. The evidence ratios of these models to the isotropic
distribution with ``flat'' spin magnitudes are 0.85, 0.39 and 0.19 for
the ``low'', ``flat'' and ``high'' spin magnitude models. Thus we
cannot rule out a mixture with the current data.

\section{Hierarchical Modelling} 
\label{sec:hierarchical}

LIGO measures $\chieff$ better than any other spin parameter, but
still with significant uncertainty, so we need to properly incorporate
measurement uncertainty in our analysis; thus our analysis must be
\emph{hierarchical} \citep{2010ApJ...725.2166H,2010PhRvD..81h4029M}.
In a hierarchical analysis, we assume that each event has a true, but
unknown, value of the effective spin, drawn from the population
distribution, which may have some parameters $\lambda$; then the
system is observed, represented by the likelihood function, which
results in a distribution for the true effective spin (and all other
parameters describing the system) consistent with the data.
Combining, the joint posterior on each system's $\chieff^i$ parameters
and the population parameters $\lambda$ implied by a set of
observations each with data $d^i$, is
\begin{equation}
  p\left( \left\{ \chieff^i \right\}, \lambda \mid \left\{ d^i \right\} \right) \propto \left[ \prod_{i=1}^{N_\mathrm{obs}} p\left(d^i \mid \chieff^i \right) p\left( \chieff^i \mid \lambda \right) \right] p\left(\lambda\right).
\end{equation}

The components of this formula are
\begin{itemize}
\item The GW (marginal) likelihood, $p\left(d \mid \chieff\right)$.
  Here we use ``marginal'' because we are (implicitly) integrating
  over all parameters of the signal but $\chieff$.  Note that it is
  the likelihood rather than the posterior that matters for the
  hierarchical analysis; if we are given posterior distributions or
  posterior samples, we need to re-weight to "remove" the prior and
  obtain the likelihood.
\item The population distribution for $\chieff$,
  $p\left( \chieff \mid \lambda \right)$.  This function can be
  parameterised by population-level parameters, $\lambda$.  (In the
  cases discussed above, there are no parameters for the population.)
\item The prior on the population-level parameters, $p(\lambda)$.
\end{itemize}
If we do not care about the individual event $\chieff$ parameters, we
can integrate them out, obtaining
\begin{equation}
  p\left( \lambda \mid \left\{ d^i \right\} \right) \propto \left[ \prod_{i=1}^{N_\mathrm{obs}} \int \dd \chieff^i \, p\left(d^i \mid \chieff^i \right) p\left( \chieff^i \mid \lambda \right) \right] p\left(\lambda\right).
\end{equation}
If we are given posterior samples of $\chieff^{ij}$ ($i$ labels the
event, $j$ labels the particular posterior sample) drawn from an
analysis using a prior $p\left( \chieff \right)$, then we can
approximate the integral by a re-weighted average of the population
distribution over the samples (here $p\left( \chieff^{ij} \right)$ is
the prior used to produce the posterior samples):
\begin{equation}
  p\left( \lambda \mid \left\{ d^i \right\} \right) \propto \left[ \prod_{i=1}^{N_\mathrm{obs}} \frac{1}{N_i} \sum_{j=1}^{N_i} \frac{p\left( \chieff^{ij} \mid \lambda \right)}{p\left( \chieff^{ij} \right)} \right] p\left(\lambda\right).
  \label{eq:hierarchical-posterior}
\end{equation}

\subsection{Order of Magnitude Calculation}
\label{sec:om-odds-ratio}

It is possible to estimate at an order-of-magnitude level the rate at
which evidence accumulates in favour of or against the isotropic
models as more systems are detected.  Based on Figure
\ref{fig:chieff-distribution-models}, approximate the isotropic
population $\chieff$ distribution as uniform on
$\chieff \in \left[ -0.25, 0.25 \right]$ and the aligned population
$\chieff$ distribution as uniform on
$\chieff \in \left[0, 0.5\right]$.  Then the odds ratio between the
isotropic and aligned models for each event is approximately
\begin{equation}
  \label{eq:approx-odds}
  \frac{p\left( d \mid I \right)}{p\left( d \mid A \right)} \simeq
  \frac{P\left( -0.25 \leq \chieff \leq 0.25 \right)}{P\left( 0 \leq \chieff \leq 0.5 \right) },
\end{equation}
where $P\left( A \leq \chieff \leq B \right)$ is the posterior
probability that $\chieff$ is between $A$ and $B$.  Using our
approximations to the $\chieff$ posteriors described above, this gives
an odds ratio of $5$ in favour of the isotropic models, which is about
a factor of two smaller than the ratio in the more careful calculation
described in Section \ref{sec:O1}.  This is a satisfactory answer at
an order-of-magnitude level.

If the true distribution is isotropic and follows this simple model,
and our measurement uncertainties on $\chieff$ are $\simeq 0.1$, then
each subsequent measurement contributes on average a factor of
$\sim 3$ to the overall odds.  After ten additional events, then, the
odds ratio becomes $5 \times 3^{10} \simeq 3 \times 10^{5}$, or
$4.6 \sigma$, consistent with the results of the more detailed
calculation described above.  If the true distribution of spins
becomes half as wide ($\chieff \in [-0.125, 0.125]$ for isotropic and
$\chieff \in [0, 0.25]$ for aligned spins), with the same
uncertainties, then the existing odds ratio becomes $1.08$, and each
subsequent event drawn from the isotropic distribution contributes on
average a factor of $1.6$.  In this case, after 10 additional events,
the odds ratio becomes $150$, or $2.7\sigma$.  With small spin
magnitudes, our angular resolving power vanishes, as discussed in more
detail in Section \ref{sec:smallspins}.

\section{Effect of small spin magnitudes}
\label{sec:smallspins}

In the main text we considered three models for BH spin magnitudes: ``low'', ``flat'' and ``high''. These were intended to capture some of the uncertainty regarding the BH spin magnitude distribution.

Here we extend the ``low'' model as:
%
\begin{equation}
p(a) \propto (1 - a)^{\alpha}
\label{eq:lowspinalpha}
\end{equation}
%

When $\alpha = 0$, this recovers the ``flat'' distribution, whilst $\alpha = 1$
recovers the ``low'' distribution. For higher values of $\alpha$, this distribution 
becomes more peaked towards $a = 0$.

%
\begin{figure}
\centering
\includegraphics[width=0.45\textwidth]{../plots/sigma_v_alpha.png}
\caption{\textbf{Effect of small spins on evidence ratio of aligned against isotropic models} The blue line shows the evidence ratio (plotted as the equivalent sigma) between a model where all systems are aligned, versus one where all systems are from an isotropic distribution as a function of the power law $\alpha$ corresponding to Equation~\ref{eq:lowspinalpha}.  The top axis shows the mean spin magnitude $a$
that the value of alpha corresponds to. We see that for mean spin magnitudes $\lesssim 0.2$ we find no evidence for either distribution over the other.}
\label{fig:smallspinsalpha}
\end{figure}
%

In Figure~\ref{fig:smallspinsalpha} we plot the evidence ratio of aligned to isotropic distributions (plotted as the equivalent sigma) with spin magnitudes given by this model with $\alpha$ in the range $0$ -- $6$. The top axis shows the mean spin magnitude
that value of alpha corresponds to (i.e., for the ``flat'' distribution $\alpha = 0$, the mean spin magnitude is 0.5). We see that if typical BH spins are $\lesssim 0.2$ we have no evidence for one model over the other.

\section{Mass Ratio}
\label{sec:mass-ratio}

Figure \ref{fig:mass-ratio-sensitivity} shows the distributions of
$\chieff$ that would obtain with a mass ratio $q = m_2/m_1 = 0.5$
compared to the distributions with $q = 1$ used above.  The details of
the distribution are sensitive to the mass ratio, but in our analysis
we are primarily sensitive to the changing \emph{sign} of $\chieff$
under the isotropic models.  This latter property is insensitive to
mass ratio.  As an example, the distinction between the three
different spin amplitude distributions after ten additional detections
is quite weak compared to the aligned/isotropic distinction in Figure
\ref{fig:O2-predictions}.  The differences in the $\chieff$
distribution between $q = 1$ and $q = 0.5$ are even smaller than the
differences between the different magnitude distributions.

\begin{figure}
  \plotone{../plots/chi-eff-distributions-q05}
  \caption{Distributions of $\chieff$ assuming all merging black holes
    have equal masses ($q=1$) or a 2:1 mass ratio ($q = 0.5$).  The
    details of the distribution are sensitive to the mass ratio, but
    in our analysis we are primarily sensitive to the changing
    \emph{sign} of $\chieff$ under the isotropic models.  This latter
    property is unchanged under changing mass ratio.}
  \label{fig:mass-ratio-sensitivity}
\end{figure}

\section{Approximations in the Gravitational Waveform}
The model waveforms used to infer the $\chieff$ of the three events
incorporate approximations to the true behaviour of the merging
systems that are expected to break down for sufficiently high
mis-aligned spins.  The effect of these approximations has been
investigated in detail for the inference on the parameters describing
GW150914 \citep{2016arXiv161107531T}.  For this source, statistical
uncertainties dominate over any waveform systematics.  Detailed
comparisons with numerical relativity computations using no
approximations to the dynamics \citep{2016PhRvD..94f4035A} also
suggest that statistical uncertainties dominate the systematics for
this system.  \citet{2016arXiv161107531T} suggests that systematics
may dominate for signals with this large SNR ($\simeq 23$) when the
source is edge-on or with high spin.  The other two events discussed
in this paper are at much lower SNR, with correspondingly larger
statistical uncertainties, and are probably similarly oriented and
with similarly small spins, so we do not expect systematic
uncertainties to dominate.  

Given the preference of our data for a smaller spin population we
assume here that measurements made in the future are not dominated by
systematic errors, but this assumption would need to be revisited for
high-SNR, edge-on, or high-spin sources detected in the future.

\bibliography{AlignedIsotropicGW}

\end{document}

